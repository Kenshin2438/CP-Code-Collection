\section{BSGS}
\cppinput{../template/number_theory/BSGS.hpp}

\section{二次剩余}
\subsection{Cipolla}
\cppinput{../template/number_theory/Cipolla.hpp}
\subsection{Tonelli Shanks}
\cppinput{../template/number_theory/Tonelli_Shanks.hpp}

\section{Ex-gcd}
得到的结果满足 $|x|+|y|$ 最小 (首要的) 同时有 $x ≤ y$ (其次)。
\cppinput{../template/number_theory/exgcd.hpp}

\section{Miller-Rabin Test}
\cppinput{../template/number_theory/miller_rabin.hpp}

\section{Sieve}
\subsection{杜教筛}
令$f(x)$为一个积性函数,求$S(n)=\sum_{i=1}^{n}f(i)$。考虑引入另一个函数$g(n)$,同时有$h(n)=f(n)*g(n)=\sum_{d \mid n}g(d)f(\frac{n}{d})$。

$$
\begin{aligned}
\sum_{i=1}^{n}{h(i)}
& = \sum_{i=1}^{n}\sum_{d \mid i}{g(d)f(\frac{i}{d})} \\
& = \sum_{d=1}^{n}{g(d)}\sum_{i=1}^{\lfloor \frac{n}{d} \rfloor}{f(i)} \\
& = \sum_{d=1}^{n}{g(d)S(\lfloor \frac{n}{d} \rfloor)} \\
\Rightarrow 
g(1)S(n) 
& = \sum_{i=1}^{n}{h(i)}-\sum_{i=2}^{n}{g(i)S(\lfloor \frac{n}{i} \rfloor)}
\end{aligned}
$$

因此,引入的函数需要满足 $\sum{h(i)},\sum{g(i)}$ 都容易求得。

\subsection{Powerful Number筛}
令$f(x)$为一个积性函数,求$S(n)=\sum_{i=1}^{n}f(i)$。考虑引入一个拟合函数$g(x)$,满足$g(p)=f(p)$,且$g(x)$为 \textbf{积性函数、前缀和易求}。

令$h=f*g^{-1}$,即有$f(n)=h(n)*g(n)$。可知:

$$f(p)=g(1)h(p)+h(1)g(p) \Rightarrow h(p)=0$$

所求的前缀和为:

$$
\begin{aligned}
S(n) 
&= \sum_{i=1}^{n}f(i) \\
&= \sum_{i=1}^{n}\sum_{d\mid i}h(d)g(\frac{i}{d}) \\
&= \sum_{d=1}^{n}h(d)\sum_{i=1}^{\lfloor\frac{n}{d}\rfloor}g(i)
\end{aligned}
$$

\begin{quotation}
Powerful Number: 由于$h(p)=0$,且$h$为积性函数,则仅当$n$满足下面的条件时,$h(n)$才有贡献。
$$n=\prod_{i=1}^{s} p_i^{t_i},\forall i \in [1, s],t_i>1$$
**关于PN的数目**,从莫比乌斯函数的角度考虑,应该为$n-\sum_{i=1}^{n}\mu^2(i)$,但是这样并不能很好的计算值。
这里用PN的一个性质,$n\in PN,\exists a,b, \mathrm{s.t. } n=a^2b^3$,则结果为$\sum_{a=1}^{\sqrt n}\sqrt[3]{\frac{n}{a^2}}$,用积分可以简单求值为$O(\sqrt n)$。
\end{quotation}  
  
\subsection{Min25筛}