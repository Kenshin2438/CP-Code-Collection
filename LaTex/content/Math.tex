\chapter{Math}
\section{Minimax搜索 alpha-beta剪枝}
\cppinput{../template/math/minimax.hpp}

\section{Combination}
% \cppinput{../template/math/combination.hpp}
\begin{cpplist}
mint C(int n, int r) {
  if (n < 0 || n < r || r < 0) return mint(0);
  return fac(n) * ifac(r) * ifac(n - r);
}
mint P(int n, int r) {
  if (n < 0 || n < r || r < 0) return mint(0);
  return fac(n) * ifac(n - r);
}
mint S2(int n, int k) {
  mint res(0);
  for (int i = 0; i <= k; i++) {
    mint t = mint(k - i).pow(n) * C(k, i);
    (i & 1) ? res -= t : res += t;
  }
  return res * ifac(k);
}
mint B(int n, int k) { // sum_{i=0}^{k} S2(n, i)
  static vector<mint> sum = { 1 };
  for (static int i = 1; i <= k; i++) {
    sum.push_back(sum.back() + (i & 1 ? -ifac(i) : ifac(i)));
  }
  mint res(0);
  for (int i = 1; i <= k; i++) {
    res += sum[k - i] * mint(i).pow(n) * ifac(i);
  }
  return res;
}
\end{cpplist}

\section{Stirling Number Query}
\cppinput{../template/math/stirling_number_module_small_p.hpp}

\section{一些数学结论}
\subsection{错排计数}
\[ D[n] = (n-1)\times(D[n-1]+D[n-2]), D[0] = 1, D[1] = 0 \]

对于第n个要加入错排的数,
\begin{itemize}
  \item 它可以和已经错排的n-1个数的其中一个进行交换,构成错排,于是有 $(n-1)\times D[n-1]$。
  \item 还可以与n-1个数中唯一一个没有加入错排的数(n-2个构成错排,剩下一个待定)进行交换,这样就构成了n错排,而这n-1个数之中,每个数都可以作为那个唯一一个没有加入错排的数,故$(n-1)\times D[n-2]$。
\end{itemize}

\subsection{Pick定理}
\[ S=a+\frac{b}{2}-1 \]

其中$a$表示多边形内部的点数,$b$表示多边形边界上的点数,$S$表示多边形的面积。

\subsection{约瑟夫环}
一共有$n$个人,每数到$k$的人离开,求第$m$个离开的人的编号(1-indexed)。
\begin{cpplist}
ll Josephus(ll n, ll m, ll k) {
  if (k == 1) return m;
  ll index = (k - 1) % (n - m + 1);
  for (ll cur = n - m + 2; cur <= n; cur++) {
    index = (index + k) % cur;
    ll tmp = min(n - cur, (cur - index - 1) / k) - 1;
    if (tmp > 0) cur += tmp, index += tmp * k;
  }
  return index + 1;
}
\end{cpplist}