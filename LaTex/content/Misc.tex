\section{Debug}
\setlength{\parskip}{0.5em}

\subsection{一些心得}
当榜上过了很多人,但是你却没思路时,试试下面这些?
\begin{itemize}
  \setlength{\itemsep}{0.5ex}
  \item 想一想{\textbf{数据范围是否有特殊意义}}。
  \item 如果是一些数学题,考虑{\textbf{打表找规律,总比死磕要好}}。
  \item 什么?是博弈?哦,两个聪明人的事,咱们不掺和。想不出就别死撑着了,{\textbf{SG函数}和\textbf{Minimax搜索}}开冲!
  \item 多翻一下带来的板子,也许是自己不会的{\textbf{人均算法}}呢?
  \item 如果你的算法复杂度比较正确(且自己已经不能再优化了),考虑玄学(Miller-Rabin / Rho 随机化等等);或者想想暴力优化?{\textbf{能过这么多,总归是的有道理}}。
  \item 选择放弃。{\textbf{就你不会写那很可能就是你太菜了,换一个题自闭去}}。
  \item 对着队友语言输出!然后把题交给队友。
\end{itemize}

\subsection{Random Number}
\cppinput{../template/debug/random_number.hpp}

\subsection{CMD对拍bat脚本}
\begin{cpplist}
@echo off

:loop
  gen.exe > _.in
  ac.exe < _.in > _.out
  bf.exe < _.in > _.ans
  fc _.out _.ans
if not errorlevel 1 goto loop
pause
goto loop
\end{cpplist}

\section{Int128}
\cppinput{../template/misc/int128.hpp}

\section{ModInt}
\cppinput{../template/misc/mint.hpp}

\section{Tree Hash}
\cppinput{../template/misc/tree_hash.hpp}