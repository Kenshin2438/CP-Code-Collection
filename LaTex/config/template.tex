%% 页面设置
\usepackage{geometry}
\geometry{a4paper, scale=0.8}
\usepackage{fancyhdr}
\pagestyle{plain}
\usepackage{enumitem} % 改善列表间距 \setlength{\itemsep}{-0.1cm}

%% 符号、字体
\usepackage{amsmath, amssymb, amsthm, bm, mathrsfs}
\usepackage{unicode-math}
\usepackage{fixdif} % 微分符号 https://zhuanlan.zhihu.com/p/521009044
\usepackage{fontspec}
\newfontface\EmojiFont{Segoe UI Emoji}%[Renderer=HarfBuzz]
\usepackage[dvipsnames]{xcolor}

%% 引入图片、绘制矢量图
\usepackage{graphicx}
\graphicspath{ 
  {./images/}
  {../images/}
}
\usepackage{caption}
\usepackage{subcaption}
\usepackage{tikz, pgf}
\usetikzlibrary{automata, positioning, arrows}

%% 代码块
\usepackage{listings}
\setmonofont{Monaco}[AutoFakeBold]
\lstnewenvironment{cpplist}{
  \lstset{
    language=C++,
    basicstyle=\footnotesize\ttfamily,
    keywordstyle=\color{purple}\bfseries,
    commentstyle=\color{gray},
    frame=lines,
    tabsize=2,
    lineskip=-1.5pt,
    extendedchars=true,
    escapeinside=``,
    breaklines=true,
    showspaces=false,
    showstringspaces=false,
    showtabs=false,
  }
}{}
\newcommand{\cppinput}[1]{
  \lstinputlisting[
    language=C++,
    basicstyle=\footnotesize\ttfamily,
    keywordstyle=\color{purple}\bfseries,
    commentstyle=\color{gray},
    frame=lines,
    tabsize=2,
    lineskip=-1.5pt,
    extendedchars=true,
    escapeinside=``,
    breaklines=true,
    showspaces=false,
    showstringspaces=false,
    showtabs=false,
  ]{#1}
}
% \usepackage{minted}
% \usemintedstyle[cpp]{murphy}
% \setminted[cpp]{
%   frame = lines,
%   framesep = 2mm,
%   baselinestretch = 1,
%   mathescape
% }

%% 链接、索引
\usepackage[colorlinks, linkcolor=black]{hyperref}
\usepackage{makeidx}
\makeindex
\bibliographystyle{plain}

%% 定理环境 https://ask.latexstudio.net/ask/article/647.html
\newtheoremstyle{mystyle}                                % 样式名
  {}                                                     % 上方间距
  {}                                                     % 下方间距
  {\normalfont}                                          % 主体字体
  {}                                                     % 缩进量
  {\bfseries}                                            % 标题字体
  {}                                                     % 标题后的点
  { }                                                    % 标题后的空白符
  {\underline{\thmname{#1}\thmnumber{#2}\thmnote{(#3)}}} % 标题样式
\theoremstyle{mystyle}

% \newtheorem{<环境名>}[<共享计数器>]{<定理头文本>}
\newtheorem{theorem}{定理}[section]
\newtheorem{lemma}{引理}[section]
\renewcommand{\proofname}{证明}

\makeatletter % use at mark
\renewenvironment{proof}[1][\proofname]{\par
  \pushQED{\qed}%
  \normalfont \topsep6\p@\@plus6\p@\relax
  \trivlist
  \item[\hskip\labelsep
        \itshape
    {\bf\underline{#1}}]\ignorespaces
    % {\bf\underline{#1}\@addpunct{.}}]\ignorespaces
}{
  \popQED\endtrivlist\@endpefalse
}
\providecommand{\proofname}{Proof}
\makeatother % end at mark